\section{Marco Teórico}
\begin{itemize}
    \item \textbf{Motor de inducción trifásico}: convierte energía eléctrica en energía mecánica. Puede conectarse en estrella (380 V) o en triángulo (220 V).
    \item \textbf{Transformador trifásico}: adapta la tensión de red (400 V) a la tensión nominal del motor (230 V).
    \item \textbf{Arranque estrella-triángulo}: método que reduce la corriente de arranque y luego conmuta a triángulo para el régimen nominal.
    \item \textbf{Tablero de comando y protección (TCP)}: contiene interruptores, guardamotores, contactores y protección diferencial para maniobrar y proteger al motor.
    \item \textbf{Circuitos}: 
    \begin{itemize}
        \item Comando: señales de baja tensión (24 V CC).
        \item Potencia: alimentación del motor (230 V CA trifásica).
    \end{itemize}
    \item \textbf{Seguridad}: se trabajó inicialmente sin tensión, evitando contacto con niveles peligrosos.
\end{itemize}
